Although Weka is a software that allows you to make automated classifications by different methods and forms, in this analysis only the methods proposed at the beginning of the article were used, in such a way that it shows, the results that were more adjusted or approximated the classifications long awaited by the expert. This is due to the fact that there are qualitative data, not all of the tools have the ability to properly classify and on the other hand the behavior cannot be explained under some statistical method. The software that best made the classifications was SALSA, see section 5.1, C and D, due to the fact that it included in the right way, the qualitative data along with the quantitative data.
With the data that has already been processed in both Weka and in SALSA, an analysis and comparison was carried out. The analysis was performed with respect to its performance to the clustering of quantitative and qualitative data, on the other hand, is made with respect to the efficiency when performing such data. In this point, it can be seen that the 3393 data 100 %, were classified properly in SALSA, unlike Weka which could not reach its classification in this same percentage.
